\likechapter{\MakeUppercase{Описание задания}}

\textbf{Постановка задачи}: Задан объект управления системой уравнений:
\begin{equation}
    \dot{x}=A x+b u, y = c x \tag{1}
\end{equation} 

\noindent где $ x=(x_1, x_2)^T $ - вектор состояния системы, $ u $ - управление, $ y $ - измерение с датчика, $ A, b, c $ - матрицы.

Требуется:
\begin{itemize}
    \item[1.] Исследовать систему (1) на управляемость и наблюдаемость.
    \item[2.] Решить задачу модальной стабилизации системы (1), выбрав в качестве характеристического многочлена системы с обратной связью стандартную форму Ньютона (для нечетного варианта):
    \begin{equation}
        p_d(\lambda)=(\lambda + 1)^2 \tag{2}
    \end{equation}
	Расчет коэффициентов обратной связи произвести по формуле Аккермана. Выполнить проверку решения.
    \item[3.] Решить линейно-квадратичную задачу оптимального управления объектом (1) с критериями качества:
    \begin{equation}
        J_1=\int^{\infty}_0 u^2 dt \rightarrow min, \; J_2=\int^{\infty}_0 (x^T Qx+u^2) dt \rightarrow min \tag{3}
    \end{equation}
    Исследовать устойчивость системы с оптимальными обратными связями.
    \item[4.] Построить асимптотический наблюдатель для оценки вектора состояния
    \begin{equation}
        \dot{\hat{x}}=A \hat{x} + b u + L(y - C \hat{x}) \tag{4}
    \end{equation} 
    \item[5.] Провести компьютерное моделирование:
    \begin{itemize}
        \item[(a)] Процессов стабилизации системы (1) по вектору состояния $ u=-kx $, для всех трёх полученных коэффициентов усиления $ k $.
        \item[(b)] Процессов стабилизации системы (1) по оценке состояния $ u=-k \hat{x} $, где $ \hat{x} $ - оценка вектора состояния, полученная асимптотическим наблюдателем. Необходимо использовать оптимальный коэффициент $ k \; (J_1 \rightarrow min) $ . 
    \end{itemize} 
\end{itemize}

\textbf{Исходные данные}:
$$
A = \begin{bmatrix}
    -1 && -1 \\ 0 && 1
\end{bmatrix}, b = \begin{bmatrix}
    0 \\ 1
\end{bmatrix}, c = \begin{bmatrix}
    1 && 1
\end{bmatrix}, Q = \begin{bmatrix}
    5 && 5 \\ 5 && 5
\end{bmatrix}
$$
