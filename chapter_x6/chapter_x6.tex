\chapter{\MakeUppercase{Компьютерное моделирование}}

Зададим наблюдатель, который будет удовлетворять найденным условиям устойчивости:
$$
L=\begin{bmatrix}
    -2 \\ 2
\end{bmatrix}
$$

Воспользуемся программой \textit{Wolfram Mathematica}. Код программы, моделирующей процессы стабилизации системы (1), находится в Приложении Б.

\textbf{1. Задача модальной стабилизации}

Смоделируем процесс стабилизации для случая, когда вектор состояния имеет вид $ u=-kx $, при этом:
$$
k=\begin{bmatrix} 0 & 2 \end{bmatrix}
$$

[график $ x_2(t) $]

\textbf{2. Линейно-квадратичная задача оптимального управления ($ J_1 $)}

Смоделируем процесс стабилизации для случая, когда вектор состояния имеет вид $ u=-kx $, при этом:
$$
k = \begin{bmatrix} 0 & 2 \end{bmatrix}
$$

[график $ x_2(t) $]

\textbf{3. Линейно-квадратичная задача оптимального управления ($ J_2 $)}

Смоделируем процесс стабилизации для случая, когда вектор состояния имеет вид $ u=-kx $, при этом:
$$
k = \begin{bmatrix} 1 & 3 \end{bmatrix}
$$

[графики $ x_1(t), x_2(t) $]

\textbf{4. Линейно-квадратичная задача оптимального управления ($ J_1 $) с ассимптотическим наблюдателем }

Смоделируем процесс стабилизации для случая, когда вектор состояния имеет вид $ u=-k\hat{x} $, при этом:
$$
k = \begin{bmatrix} 0 & 2 \end{bmatrix}
$$

[график $ x_2(t) $]
