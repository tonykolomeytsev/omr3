\chapter{\MakeUppercase{Решение линейно-квадратичной задачи оптимального управления}}

Решим линейно-квадратичную задачу оптимального управления объектом (1) с критерием качества:
$$
J_1=\int^{\infty}_0 u^2 dt \rightarrow min
$$

\noindent Система вполне управляема. Оптимальное управление имеет вид:
\begin{align*}
    &u=-kx \\
    &k=B^T P
\end{align*}

\noindent где P-решение алгебраического уравнения Риккати. Ниже уравнение Рикатти:
$$
PA + A^T P - PBB^T P=0
$$

Пусть $ P = \begin{bmatrix} p_{11} & p_{12} \\ p_{21} & p_{22} \end{bmatrix} $, тогда:
\begin{align*}
    &PA = \begin{bmatrix}
        -p_{11} & -p_{11}+p_{12} \\ -p_{21} & -p_{21}+p_{22}
    \end{bmatrix} \\
    & A^T P = (PA)^T = \begin{bmatrix}
        -p_{11} & -p_{21} \\ -p_{11}+p_{12} & -p_{21}+p_{22}
    \end{bmatrix} \\
    &PB = \begin{bmatrix} p_{11} & p_{12} \\ p_{21} & p_{22} \end{bmatrix} \begin{bmatrix} 0 \\ 1 \end{bmatrix} = \begin{bmatrix} p_{12} \\ p_{22} \end{bmatrix} \\
    &B^T P=\begin{bmatrix} p_{12} & p_{22} \end{bmatrix} \\
    &PBB^T P = \begin{bmatrix} p_{12}^2 & p_{12}p_{22} \\ p_{12}p_{22} & p_{22}^2 \end{bmatrix}
\end{align*}

\noindent Полученное уравнение Рикатти:
$$
    \begin{bmatrix}
        -p_{11} & -p_{11}+p_{12} \\ -p_{21} & -p_{21}+p_{22}
    \end{bmatrix} + \begin{bmatrix}
        -p_{11} & -p_{21} \\ -p_{11}+p_{12} & -p_{21}+p_{22}
    \end{bmatrix} - \begin{bmatrix} p_{12}^2 & p_{12}p_{22} \\ p_{12}p_{22} & p_{22}^2 \end{bmatrix}
$$
\noindent поэлементная запись которого выглядит следующим образом:
$$
    \begin{cases}
        -p_{12}^2-2 p_{11} = 0 \\
        -2 p_{21}-\left(p_{22}-2\right) p_{22} = 0 \\
        -p_{11}-p_{21}-p_{12} \left(p_{22}-1\right) = 0
    \end{cases}
$$

\noindent Решая систему уравнений получим следующие значения: $ p_{11}= 0, p_{12} = 0,p_{22} = 2 $, и значит матрица $ P $ выглядит следующим образом:
$$ 
P = \begin{bmatrix}
 0 & 0 \\
 0 & 2 \\
\end{bmatrix}
$$

\noindent Определитель $ \det P = 0 $, позже проверим ассимптотическую устойчивость. Составим оптимальное уравнение:
\begin{align*}
    &k=B^T P = \begin{bmatrix}
        0 & 2
    \end{bmatrix} \\
    &U_{opt}=0x_1+2x_2
\end{align*}

\textbf{Выполним проверку}

При $ u=-kx $ уравнения состояния имеют вид:
$$ \dot{x}=Ax+bu=Ax-bkx=(A-bk)x $$

\noindent Заново вычислим характеристический полином системы:
\begin{align*}
    &A-bk = \begin{bmatrix}
        -1 & -1 \\ 0 & 1
    \end{bmatrix} - \begin{bmatrix}
        0 \\ 1 
    \end{bmatrix} \begin{bmatrix}
        0 & 2
    \end{bmatrix} =\begin{bmatrix}
        -1 & -1 \\ 0 & -1
    \end{bmatrix} \\
    &| \lambda E - (A-bk) | = \begin{vmatrix}
        1+\lambda & 1 \\ 0 & 1+\lambda
    \end{vmatrix} = \lambda:2 + 2\lambda + 1
\end{align*}

\noindent по критерию Гурвица система с найденным управлением является ассимптотически утойчивой.


Решим линейно-квадратичную задачу оптимального управления объектом (1) с критерием качества:
$$
J_2=\int^{\infty}_0 (x^T Qx+u^2) dt \rightarrow min
$$


