\chapter{\MakeUppercase{ Исследование системы на управляемость и наблюдаемость }}
\textbf{Проверка на управляемость}

Так как матрицы $ A $ и $ b $ константны, можно использовать критерий Калмана:
\begin{align*}
    &\mathbb{C} (A, b) = [b | Ab] \\
    &Ab =\begin{bmatrix}
        -1 & -1 \\ 0 & 1
    \end{bmatrix} \begin{bmatrix}
        0 \\ 1
    \end{bmatrix} = \begin{bmatrix}
        -1 \\ 1
    \end{bmatrix} \\
    &\mathbb{C} (A, b) = \begin{bmatrix}
        0 & -1 \\ 1 & 1
    \end{bmatrix}
\end{align*}
\noindent Определитель матрицы Калмана: $ \det\mathbb{C} = 1 \neq 0 $. Определитель не равен нулю, а это значит ранг матрицы Калмана равен 2, следовательно система вполне управляема.

\textbf{Проверка на наблюдаемость}

Вектор измерений: $ y=cx $. Используем критерий Калмана для матриц $ A, c $, так как они константны:
\begin{align*}
    & \mathbb{O}(A, c) = \begin{bmatrix}
        c \\ cA
    \end{bmatrix} \\
    & cA = \begin{bmatrix}
        1 & 1
    \end{bmatrix} \begin{bmatrix}
        -1 & -1 \\ 0 & 1
    \end{bmatrix} = \begin{bmatrix}
        -1 & 0
    \end{bmatrix} \\
    & \mathbb{O}(A, c) = \begin{bmatrix}
        1 & 1 \\
        -1 & 0
    \end{bmatrix}
\end{align*}
\noindent Значение определителя $ \det\mathbb{O} = 1 \neq 0 $/ Определитель не равен нулю, следовательно система вполне наблюдаема и можно точно оценить вектор состояния. 